\documentclass[12pt,a4paper,AutoFakeBold]{article}
\usepackage{gzhubenke}

\begin{document}
\pagenumbering{gobble} 

%% 封面部分
\begin{titlepage}
\vspace*{0.15cm}

\begin{figure}[H]
	\centering
	\includegraphics[scale=1.25]{assets/gzhu.png}
\end{figure}
\vspace{0.15cm}	
\centering

{\zihao{1}\kai \textbf{本~~~科~~~毕~~~业~~~论~~文~~~(设计)}}

\vspace{2.5cm}

% \begin{picture}(0,0)
%     \put(-220,-150){\zihao{3}\kai 教}
%     \put(-220,-185){\zihao{3}\kai 务}
%     \put(-220,-220){\zihao{3}\kai 处}
%     \put(-220,-255){\zihao{3}\kai 制}
% \end{picture}

\begin{flushleft}
	\hspace*{-6em}
	\begin{minipage}{\linewidth}
		{{\kai \zihao{-2} \qquad\qquad\qquad 课题名称}\quad{\zihao{3} \kai
        \begin{minipage}[b]{9.5cm}
        \centering
        % 太长这样换行
        % {这里是课题名称}
        \underline{\hbox to 9.5cm{\hfil XZXXXXXXXX\hfil}}
        \end{minipage}}\par}
		\vspace{0.5cm}
		{{\kai \zihao{-2} \qquad\qquad\qquad 学\qquad 院}\quad{\zihao{3} \kai \dlmu[9.5cm]{这里是学院}}\par}
		\vspace{0.5cm}
		{{\kai \zihao{-2} \qquad\qquad\qquad 专\qquad 业}\quad{\zihao{3} \kai\dlmu[9.5cm]{这里是专业}}\par}
		\vspace{0.5cm}
		{{\kai \zihao{-2} \qquad\qquad\qquad 班级名称}\quad{\zihao{3} \kai\dlmu[9.5cm]{      }}\par}
		\vspace{0.5cm}
		{{\kai \zihao{-2} \qquad\qquad\qquad 学生姓名}\quad{\zihao{3} \kai\dlmu[9.5cm]{     }}\par}
		\vspace{0.5cm}
		{{\kai \zihao{-2} \qquad\qquad\qquad 学\qquad 号}\quad{\zihao{3} \kai\dlmu[9.5cm]{     }}\par}
		\vspace{0.5cm}
		{{\kai \zihao{-2} \qquad\qquad\qquad 指导老师}\quad{\zihao{3} \kai\dlmu[9.5cm]{    }}\par}
		\vspace{0.5cm}
		{{\kai \zihao{-2} \qquad\qquad\qquad 完成日期}\quad{\zihao{3} \kai\dlmu[9.5cm]{      }}\par}
	\end{minipage}
\end{flushleft}

\vspace{3cm}
{\kai \zihao{3}{教\quad 务\quad 处\quad 制}}

\vspace{4cm}
\end{titlepage}

% 摘要与关键词
% \pagenumbering{Roman}  % 使用罗马数字页码
% \setcounter{page}{1}   % 页码从1开始
{ 
\mkls
\renewcommand{\abstractname}{\scriptsize}
% \phantomsection
% \addcontentsline{toc}{section}{摘要} 
%% 摘要和关键词部分
\begin{center}
	{\hei\zihao{3}\textbf{这里是论文标题}}\par
	{\zihao{-4}\song XXXXXXXXXX ~~ 专业 ~~ XXXXX班 ~~ XXX \par 
		指导教师:XXXX
	}
\end{center}


\begin{onecolabstract}
\noindent{}{\zihao{4}\textbf{摘要\qquad}}{\song \zihao{-4}{
	这是摘要
}
}\par

\vspace{22pt}
\noindent{}{\zihao{4}\textbf{关键词\qquad}}{\zihao{-4}\song {
关键字1,关键字2
} }\par
\end{onecolabstract}

\newpage
\begin{onecolabstract}
% \phantomsection
% \addcontentsline{toc}{section}{ABSTRACT} 
\noindent{}{\zihao{4} \textbf{ABSTRACT\qquad}}{\zihao{-4}{
Abstract here
}}\par
	
% 英文关键词
\vspace{22pt}
\noindent{}{\zihao{4}\textbf{KEYWORDS\qquad}}{\zihao{-4}{
Keyword1, Keyword2
}}\par
\end{onecolabstract}
}


\newpage
% \phantomsection
% \addcontentsline{toc}{section}{目录} 
%% 目录部分
\renewcommand{\contentsname}{\centerline{\zihao{-2}\textbf{目录}}}
\linespread{1.2}\selectfont{
	\tableofcontents
}

\newpage

%% 下面是正文
\pagenumbering{arabic}  % 使用阿拉伯数字页码
\setcounter{page}{1}  
{
% \setlength{\baselineskip}{23pt} 
\mkls% 设置行间距
\song% 设置宋体
\zihao{-4} % 设置小四号字体

\setcounter{section}{0} % 重置章节计数器
\section{前言}
\label{sec:introduction}
\subsection{研究背景与意义}
\label{sec:background}
很有意义
\subsection{国内外研究现状}
很有研究,比如\cite{beckSurveyMetaReinforcementLearning2023}
\subsection{本文的主要贡献}
没什么贡献,如\refig{fig:gzhu}所示

\subsection{论文组织结构}

\begin{figure}[htbp] % use H to force the figure to be placed here
	\centering
	\includegraphics[width=0.8\textwidth]{assets/gzhu.png}
	\caption{图片}
	\label{fig:gzhu}
\end{figure}

\section{提供了一些简单的宏}

使用`\verb|\refsec{sec:background}|`可以引用\refsec{sec:introduction},使用`\verb|\refsubsec{sec:background}|`可以引用\refsubsec{sec:background},使用`\verb|\refig{fig:gzhu}|`可以引用\refig{fig:gzhu},使用`\verb|\refeq{eq:esv}|`可以引用公式\refeq{eq:esv}。

\begin{equation}
    \mathbf{e}(\phi,\bar{M}) = [1, e^{j \pi \phi},...,e^{j (\bar{M} - 1) \pi \phi}]^T 
    \label{eq:esv}
\end{equation}

\newpage
%参考文献
\renewcommand\refname{参考文献}
\label{sec:references}
\phantomsection
\addcontentsline{toc}{section}{参考文献}
\bibliography{references}


\newpage
\phantomsection
\section*{致 \quad 谢}
\addcontentsline{toc}{section}{致 谢}
我真的谢谢
\par
\vspace{5ex}
\rightline{\zihao{3}{XXXX\quad\qquad}}
\rightline{二O二五年五月二日于广州}

}

\end{document}
