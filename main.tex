\documentclass[12pt,a4paper,AutoFakeBold]{article}
\usepackage{graphicx}
\usepackage[fontset=none,UTF8,heading=true]{ctex}
% \usepackage{CJK}
\usepackage{indentfirst}
%\graphicspath{{chapter/}{figures/}}
\usepackage{amsmath}%数学

\usepackage{fontspec}
\setmainfont{Times New Roman} % 设置默认英文字体为 Times New Roman
\setsansfont{Times New Roman} % 设置无衬线英文字体
\setmonofont{Times New Roman} % 设置等宽英文字体

\setCJKmainfont[BoldFont={SimHei}]{SimSun} % 默认使用宋体,加粗时使用黑体
\setCJKsansfont{SimHei} % 设置无衬线字体为黑体
\setCJKfamilyfont{song}{SimSun} % 设置宋体
\setCJKfamilyfont{hei}{SimHei} % 设置黑体
\setCJKfamilyfont{kai}{KaiTi_GB2312} % 设置楷体

\newcommand{\hei}{\CJKfamily{hei}}
\newcommand{\kai}{\CJKfamily{kai}}
\newcommand{\song}{\CJKfamily{song}}

%\usepackage[colorlinks,linkcolor=red]{hyperref}%超链接

\usepackage{fancyhdr}  %使用fancyhdr包自定义页眉页脚
%\pagestyle{empty}
% \pagestyle{fancy}
\pagestyle{plain}%没有页眉,页脚放页数

\usepackage{titlesec}%设置章节标题与正文间距为2行
\titlespacing{\section}{0pt}{0pt}{2em}

\usepackage{enumerate}%项目编号

\renewcommand{\figurename}{图}%将figure改为图
\renewcommand{\tablename}{表}

\usepackage{caption}  % 使用更现代的caption包
\DeclareCaptionFormat{myformat}{\song\zihao{5}#1#2\quad#3}
\captionsetup{format=myformat, labelsep=none}
% \renewcommand{\captionlabeldelim}{}
\renewcommand{\thetable}{\thesection{}.\arabic{table}}%表格索引该为按照章节
\renewcommand {\thefigure} {\thesection{}.\arabic{figure}}%图片索引该为按照章节

\renewcommand{\headrulewidth}{0.5pt}
\renewcommand{\footrulewidth}{0.4pt}
\lhead{}
\chead{}
\rhead{}
\lfoot{}
\cfoot{\thepage}
\rfoot{}
\newcommand{\mkls}{\linespread{1.7}\selectfont}

% 为section 添加引导线
\usepackage{tocloft}
\renewcommand{\cftsecleader}{\cftdotfill{\cftdotsep}}


\usepackage{booktabs}%表格用

\usepackage{titlesec}%修改标题格式宏包
% 设置section标题为三号黑体居中
\titleformat{\section}{\centering\zihao{3}\hei\bfseries}{第\chinese{section}章}{0.5em}{}
% 设置subsection标题为小四号黑体
\titleformat{\subsection}{\zihao{-4}\hei\bfseries}{\arabic{section}.\arabic{subsection}}{0.5em}{}
% 设置subsubsection标题为小四号黑体
\titleformat{\subsubsection}{\zihao{-4}\hei\bfseries}{\arabic{section}.\arabic{subsection}.\arabic{subsubsection}}{0.5em}{}

\usepackage{multirow}%跨行表格
\usepackage{abstract}%摘要
\usepackage{setspace}   %行间距的宏包
\usepackage{makecell}%表格竖线连续

\def\I{\vrule width1.2pt}
%!\I 就可以代替| 来画表格了

%可固定下划线长度
\makeatletter
\newcommand\dlmu[2][4cm]{\hskip1pt\underline{\hb@xt@ #1{\hss#2\hss}}\hskip3pt}
\makeatother

\usepackage{float}%可以用于禁止浮动体浮动

%目录超链接
\usepackage[colorlinks,linkcolor=black,anchorcolor=blue,citecolor=black]{hyperref}

\usepackage{listings}%可以插入代码
\usepackage{xcolor}%语法高亮支持

%代码格式
\definecolor{dkgreen}{rgb}{0,0.6,0}
\definecolor{gray}{rgb}{0.5,0.5,0.5}
\definecolor{mauve}{rgb}{0.58,0,0.82}
\usepackage{fontspec}
\setmonofont{Consolas}
\lstset{ %
	numbers=left, 
	basicstyle=\tiny\ttfamily, 
	numberstyle=\tiny, 
	tabsize=4,
	numbersep=5pt, 
	keywordstyle= \color{blue!70}, %关键词为蓝色
	commentstyle=\color{gray}, %注释为灰色
	frame=shadowbox, % 框架阴影效果
	rulesepcolor= \color{ red!20!green!20!blue!20} ,
	escapeinside={\%*}{*)},
	xleftmargin=2em, % 边界选项
	xrightmargin=2em, % 边界选项
	aboveskip=1em, % 边界选项
	framexleftmargin=2em, % 边界选项
	breaklines,%过长代码自动换行
}

%目录标题间距
\setlength{\cftaftertoctitleskip}{3em}

%设置页面格式
\usepackage[left=3.0cm, right=2.6cm, top=2.54cm, bottom=2.54cm]{geometry}
\usepackage{setspace}

\usepackage{gbt7714}%参考文献格式 https://github.com/zepinglee/gbt7714-bibtex-style/tree/master
\bibliographystyle{gbt7714-numerical}

%% 设置节号样式
\ctexset{
    section = {
        format+ = \raggedright\hei\zihao{3}, % 设定字体格式
        name = {第,章},  % 定义章节名称格式,如“第一章”
        number = \chinese{section}  % 章节号使用中文数字
    }
}

\begin{document}


%% 封面部分
\begin{titlepage}
\vspace*{0.15cm}

\begin{figure}[H]
	\centering
	\includegraphics[scale=1.25]{logo/gzhu.png} %1.png是图片文件的相对路径
\end{figure}
\vspace{0.15cm}	
\centering

{\zihao{1}\kai \textbf{本~~~科~~~毕~~~业~~~论~~文~~~(设计)}}

\vspace{2.5cm}

\begin{picture}(0,0)
    % 使用堆叠的方式显示竖排文字
    \put(-220,-150){\zihao{3}\kai 教}
    \put(-220,-185){\zihao{3}\kai 务}
    \put(-220,-220){\zihao{3}\kai 处}
    \put(-220,-255){\zihao{3}\kai 制}
\end{picture}

\begin{flushleft}
	\hspace*{-6em}
	\begin{minipage}{\linewidth}
		{{\kai \zihao{-2} \qquad\qquad\qquad 课题名称}\quad{\zihao{3} \kai \dlmu[9.5cm]{latex模板1234}}\par}
		\vspace{0.5cm}
		{{\kai \zihao{-2} \qquad\qquad\qquad 学\qquad 院}\quad{\zihao{3} \kai \dlmu[9.5cm]{XXXXXX学院}}\par}
		\vspace{0.5cm}
		{{\kai \zihao{-2} \qquad\qquad\qquad 专\qquad 业}\quad{\zihao{3} \kai\dlmu[9.5cm]{XXXXX}}\par}
		\vspace{0.5cm}
		{{\kai \zihao{-2} \qquad\qquad\qquad 班级名称}\quad{\zihao{3} \kai\dlmu[9.5cm]{XXXXXXX}}\par}
		\vspace{0.5cm}
		{{\kai \zihao{-2} \qquad\qquad\qquad 学生姓名}\quad{\zihao{3} \kai\dlmu[9.5cm]{这是人名}}\par}
		\vspace{0.5cm}
		{{\kai \zihao{-2} \qquad\qquad\qquad 学\qquad 号}\quad{\zihao{3} \kai\dlmu[9.5cm]{000009000000}}\par}
		\vspace{0.5cm}
		{{\kai \zihao{-2} \qquad\qquad\qquad 指导老师}\quad{\zihao{3} \kai\dlmu[9.5cm]{这也是人名}}\par}
		\vspace{0.5cm}
		{{\kai \zihao{-2} \qquad\qquad\qquad 完成日期}\quad{\zihao{3} \kai\dlmu[9.5cm]{*************}}\par}
	\end{minipage}
\end{flushleft}

\vspace{4cm}
\end{titlepage}

% 摘要与关键词
{ 
\mkls
\renewcommand{\abstractname}{\scriptsize}
%% 摘要和关键词部分
\begin{center}
	{\hei\zihao{3}\textbf{*************系统设计}}\par
	{\zihao{-4}\song XXXX ~~ 专业 ~~ XXXX班 ~~ 这是人名 \par 
		指导教师:这也是人名
	}
\end{center}


\begin{onecolabstract}
\noindent{}{\zihao{4}\textbf{摘要\qquad}}{\song \zihao{-4}{
摘要内容(小四号宋体)论文摘要以浓缩的形式概括研究课题的内容,中文摘要在200字左右。X X X X X X X X X X X X X X X X X X X X X X X X X X X X X X X X X X X X X X X X X X X X X X X X X X X X X X X X X X X X X X X X X X X X X X X X X X X X X X X X X X X X X X X X X X X X X X X X X X X X X X。
		
X X X X X X X X X X X X X X X X X X X X X X X X X X X X X X X X X X X X X X X X X X X X X X X X X X X X X X X X X X X X X X X X X X X X X X X X X X X X X X X X X X X X X X X X X X X X X X X X X X X X X X。		
}
}\par

\vspace{22pt}
\noindent{}{\zihao{4}\textbf{关键词\qquad}}{\zihao{-4}\song {
关键词1;关键词2
} }\par
\end{onecolabstract}


\begin{onecolabstract}
\noindent{}{\zihao{4} \textbf{ABSTRACT\qquad}}{\zihao{-4}{
(四号 Times New Roman  字体加粗,空两格)This is abstract in English外文摘要以200个左右实词为宜,摘要内容(小四号Times New Roman字体)每段开头留四个空字符X X X X X X X X X X X X X X X X X X X X X X X X X X X X X X X X X X X X X X X X X X X X X X X X X X X X X X X X X X X X X X X X X X X X X X X X X X X X X X X X X X X X X X X X X X X X X X X X X X X X X X。

X X X X X X X X X X X X X X X X X X X X X X X X X X X X X X X X X X X X X X X X X X X X X X X X X X X X X X X X X X X X X X X X X 

}}\par
	
% 英文关键词
\vspace{22pt}
\noindent{}{\zihao{4}\textbf{KEY WORDS\qquad}}{\zihao{-4}{
keywords1; keywords2
}}\par
\end{onecolabstract}
}


\newpage


%% 目录部分
\renewcommand{\contentsname}{\centerline{\zihao{-2}\textbf{目录}}}
\linespread{1.2}\selectfont{
	\tableofcontents
}

\newpage

%% 下面是正文!
{
% \setlength{\baselineskip}{23pt} 
\mkls% 设置行间距
\song% 设置宋体
\zihao{-4} % 设置小四号字体

\setcounter{section}{0} % 重置章节计数器
\section{绪论}


\subsection{课题背景}

前景很好
\cite{beckSurveyMetaReinforcementLearning2023}
\subsection{国内外研究进展情况}
非常不多见

\begin{figure}
	\centering
	\includegraphics[width=0.8\textwidth]{logo/gzhu.png}
	\caption{这是一张图片}
	\label{fig:1}
\end{figure}

\subsection{本课题的目的及意义}
本课题的创新点是让没写过论文的人写论文

\subsection{总体研究思路}

总体的方向是抓住方向的总体

\begin{table}[htbp]
	\centering
	\caption{不同算法的性能比较}
	\begin{tabular}{|c|c|c|c|c|}
	  \hline
	  \multirow{2}{*}{\textbf{算法}} & \multicolumn{2}{c|}{\textbf{时间复杂度}} & \multicolumn{2}{c|}{\textbf{空间复杂度}} \\
	  \cline{2-5}
	  & 平均情况 & 最坏情况 & 平均情况 & 最坏情况 \\
	  \hline
	  快速排序 & $O(n\log n)$ & $O(n^2)$ & $O(\log n)$ & $O(n)$ \\
	  \hline
	  归并排序 & $O(n\log n)$ & $O(n\log n)$ & $O(n)$ & $O(n)$ \\
	  \hline
	  堆排序 & $O(n\log n)$ & $O(n\log n)$ & $O(1)$ & $O(1)$ \\
	  \hline
	  \multicolumn{5}{|c|}{\textbf{高级算法}} \\
	  \hline
	  基数排序 & \multicolumn{2}{c|}{$O(nk)$} & \multicolumn{2}{c|}{$O(n+k)$} \\
	  \hline
	\end{tabular}
	\label{tab:algorithm_comparison}
\end{table}


% 以下是结论
\newpage
\phantomsection
\section*{结 \quad 论}
\addcontentsline{toc}{section}{结\quad 论}
\newcounter{结论编号}   %创建一个计数器,这个计数用于给结论章节编号                     
\setcounter{结论编号}{\value{section}} %计数器就像变量一样
\addtocounter{结论编号}{1}


\newpage
\phantomsection
\section*{致 \quad 谢}
% \addcontentsline{toc}{section}{致谢\tiny{\quad.\quad.\quad.\quad.\quad.\quad.\quad.\quad.\quad.\quad.\quad.\quad.\quad.\quad.\quad.\quad.\quad.\quad.\quad.\quad.\quad.\quad.\quad.\quad.\quad.\quad.\quad.\quad.\quad.\quad.\quad.\quad.\quad.\quad.\quad.\quad.\quad.\quad.\quad.\quad.\quad.\quad.\quad.\quad.\quad.\quad.\quad}}
\addcontentsline{toc}{section}{致\quad 谢}
我真的谢谢你
\par
\vspace{5ex}
\rightline{\zihao{3}{这是人名\quad\qquad}}
\rightline{二O一九年五月十九日于广州}
\newpage
}



%参考文献
\renewcommand\refname{参考文献}
\label{sec:references}
\phantomsection
\addcontentsline{toc}{section}{参考文献}
\bibliography{references}

\end{document}
